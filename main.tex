\documentclass{article}
\input{structure.tex} % Include the file specifying the document structure and custom commands

% ----------------------------------------------------------------------------------------
% TITOLO
% ----------------------------------------------------------------------------------------
\title{Deadlock Detection in a Distributed System}
\author{Rocco Lo Russo, Agostino D'Amora \\ \texttt{https://github.com/h-tajato}\\\texttt{htajato@gmail.com}}
\date{Università degli studi di Napoli\\"Federico II"}

% ----------------------------------------------------------------------------------------
\begin{document}


\maketitle

\section*{Introduzione}
In un sistema distribuito, i vari nodi all'interno del sistema effettuano delle valutazioni proprie. Tali valutazioni possono essere:
\begin{itemize}
    \item Calcolo della media
    \item Calcolo della moda
    \item Calcolo del livello di confidenza (include deviazione standard [varianza])
\end{itemize}

Tali informazioni vengono poi mostrate nel diplay. Nel sistema le richieste dei dati sono bloccanti, tali richieste, quindi, possono essere soggette a possibili deadlock. Quindi, i dispositivi, dato un timeout, decidono quando avviare l'algoritmo di snapshot, per effettuare una deadlock detection e di conseguenza, riportare il sistema in un regime di funzionamento opportuno

\section{Specific}
\begin{itemize}
    \item Il sistema è asincrono (precisamente, semi-sincrono)
    
    \item Processi non soggetti a fallimenti
    
    \item Canali affidabili, unidirezionali e fifo (attenzione al lato implementativo e al wireless)
    
    \item Ogni processo può comunicare con tutti gli altri in maniera diretta
    
    \item Ogni processo può far partire una deadlock detection
    
    \item Ogni processo continua la sua normale esecuzione durante l'algoritmo
\end{itemize}

Applicazione:
\begin{itemize}
    \item Leggere i valori del sensore di temperatura
    \item Fai N-1 Richieste di temperatura (una alla volta, da considerare bloccanti)
    \item  Calcolo dei valori (media, moda, ecc.) e mosta a schermo
    \item Timeout (sia sulle richieste [Data dalle ipotesi]) per algoritmo di snapshot + deadlock detection
\end{itemize}

Sistema:
\begin{itemize}
    \item Comunicazione di rete rcv bloccante (pop dalla testa della coda)
    \item Comunicazione di send affidabile
    \item Gestione delle code di ricezione
    \item Gestione del timeout con la rcv
    \item Gestione del gruppo (autenticazione, xor time)
\end{itemize}

\section{Impostazione Tecnica}

\section{Testing e conclusioni}

% TAVOLA DEI CONTENUTI


\end{document}